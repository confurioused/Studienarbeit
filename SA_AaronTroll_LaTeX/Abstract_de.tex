\section{Abstrakt}
Periodische Beinbewegungen im Schlaf sind Symptome verschiedener Krankheitsbilder, welche in einem Polysomnogramm aufgezeichnet werden können. Die Anzahl der periodischen Beinbewegungen (PLM) pro Stunde (PLM-Index) kann einen Hinweis darauf geben, wie sehr der Schlaf gestört wird. Um den PLM-Index automatisch zu bestimmen, werden Eventdetektoren eingesetzt. 

Diese Arbeit untersucht zunächst theoretisch die Vergleichbarkeit der Eventdetektoren und zeigt, dass die bisher verwendeten Metriken nur begrenzt zur Bewertung verwendet werden können. Das hier vorgeschlagene Kostenfunktional basiert darauf, die Fehler aufzusummieren, die beim Bestimmen des PLM-Indexes entstanden sind. 
Zusätzlich werden Möglichkeiten vorgeschlagen, wie die Aussagekraft des Kostenfunktionals über den Detektor erhöht und weitere Information über den Detektor aus den Annotationssignalen extrahiert werden kann.
Die gefundenen Ansätze werden an einem Detektor aus der Literatur \cite{Moore} unter Verwendung eines Datensatzes des Uniklinikums Dresden (\ref{chap:Datensatz}) überprüft und ausgewertet. 
