\chapter{Einleitung}\label{chap:Einleitung}
Periodische Beinbewegungen, die im Schlaf auftreten, sind Symptome verschiedener Krankheitsbilder. Sie können den Schlaf stören und werden deshalb zu Therapie- und Diagnosezwecken im Schlaflabor quantifiziert. Die Auswertung der aufgenommenen Polysomnogramme wird manuell von medizinischem Personal durchgeführt. 
Dies dauert bei Experten pro ausgewertete Nacht circa zwei bis vier Stunden \cite{Huang}. Insbesondere bei monotonen Aufgaben treten dabei vermehrt Fehler auf. Es werden \glspl{LM} vor allem dann übersehen, wenn viele Events auftreten \cite{worldsleepcongress}. Besonders im späteren Verlauf der Nacht und in Schlafstadien, in denen keine \glspl{LM} erwartet werden \cite{Huang}, wird weniger zuverlässig gearbeitet. Eine Variabilität im Datensatz entsteht auch durch Subjektivität und den allgemeinem Gemütszustand des medizinischen Personals \cite{worldsleepcongress}. Eine ganz oder teilweise Automation würde die Annotationen vereinheitlichen und beschleunigen. Das Arbeiten mit halbautomatischen Annotationsunterstützungen ist 2.41 \cite{interscorer} bis 2.8 \cite{Roessen} mal schneller. 

Einige Algorithmen zum Erkennen von Beinbewegungen (Detektoren) wurden in der Literatur bereits vorgeschlagen \cite{Huang,Moore,Carvelli}. Im Idealfall arbeitet ein Detektor gut genug, sodass die verantwortlichen Entscheidungsträger dem System vertrauen und die Ergebnisse nicht einzeln überprüft werden müssen. Die Kaufentscheidung wird in manchen Fällen von Verwaltungskräften getroffen und nicht von Schlafspezialisten \cite{SleepDisordersMedicine}. Diese Verwaltungskräfte brauchen also eine einfach verstehbare Möglichkeit die Detektoren zu vergleichen, um entscheiden zu können, welches System gekauft werden soll. 

Derzeit werden Detektoren entwickelt und veröffentlicht, in denen die Autoren verschiedene Metriken angeben. Da immer mehrere der klassischen Metriken angegeben werden, widersprechen diese sich teilweise untereinander in der Aussage zur Güte des Detektors. Auch die Berechnungsweise der Metriken unterscheidet sich in den Veröffentlichungen. Für neu entwickelte Detektoren lässt sich also nicht immer eindeutig entscheiden, ob der Stand der Technik verbessert werden konnte.
Ziel dieser Arbeit ist es, Metriken zu identifizieren, welche eine fundiertere Vergleichbarkeit der Detektoren ermöglichen. 

Zuerst wird der medizinische Kontext der periodischen Beinbewegungen erläutert, um in die Thematik einzuführen und medizinisch relevante Kenngrößen zu identifizieren.
Anschließend wird erklärt, wie die Messwerte aus dem Schlaflabor zu den medizinischen Kenngrößen weiterverarbeitet werden und wie Detektoren aus der Literatur derzeit verglichen werden. 
Hieraus ergibt sich das Ziel und die Notwendigkeit dieser Arbeit im Kapitel \ref{chap:Präzisierung_der_Aufgabenstellung}.
Im Hauptteil dieser Arbeit sollen zuerst die Lösungsansätze theoretisch hergeleitet werden und anschließend praktisch getestet werden. Für die Anwendung der neuen Metriken soll ein Detektor aus der Literatur ausgewählt werden und dieser mithilfe eines Datensatzes bewertet werden. Der Datensatz stammt aus dem Uniklinikum Dresden und beinhaltet 6216 polysomnographische Aufzeichnungen.
